\documentclass{article}
\usepackage[utf8]{inputenc}
\usepackage[T2A]{fontenc}
\usepackage[russian]{babel}
\usepackage{amsmath}
\usepackage{amssymb}
\usepackage{tikz}
\usepackage{geometry}
\usepackage{tikz}
\thispagestyle{empty}
\geometry{a4paper, left=0cm, right=1cm, top=2cm, bottom=2cm}
\begin{document}
    \begin{minipage}[t]{0.25\textwidth}
        \begin{tikzpicture}
            \draw[red, line width=3pt] (0,0) circle (4cm);
            \draw[blue, line width=3pt] (-60:1.5) circle (2.5cm);
            
            \filldraw[red] (0,0) circle (1pt) node[above left, text=black] {$D$};
            \filldraw[red] (-10:4) circle (1pt) node[above right, text=black] {$F$};
            \filldraw[red] (-10:1.5) circle (1pt) node[above right, text=black] {$A$};
            \filldraw (-60:1.5) circle (1pt) node[below left] {$B$};
            \filldraw (-150:2) circle (1pt) node[below left] {$C$};
            \filldraw (-60:4) circle (1pt) node[below right] {$E$};
        
            %Линии
            \draw[red, line width=3pt] (0,0) -- (-10:1.5);
            \draw[blue, line width=3pt] (-10:1.5) -- (-10:4);
            \draw[red, line width=3pt, densely dashed] (0,0) -- (-60:1.5);
            \draw[black, line width=3pt, densely dashed] (-10:1.5) -- (-60:1.5);
            \draw[black, line width=3pt] (-150:2) -- (-60:1.5);
            \draw[orange, line width=3pt] (-60:1.5) -- (-60:4);
        \end{tikzpicture}
    \end{minipage}
    \hfill
    \begin{minipage}{0.6\textwidth}
        \raggedright
        \Large
        26 \qquad \textbf{КНИГА I ПРЕДЛ. II. ЗАДАЧА}
        \vspace{1cm}
        
        { \includegraphics[width=0.2\linewidth]{Letter_O.png}}
        \begin{minipage}{0.75\textwidth}
            \raggedright 
            \vspace*{-3cm}
            \textit{т данной точки} 
            \begin{tikzpicture}
                \node at (0,0.2) {\small A};
                \draw[line width=3pt, red] (-0.7,0) -- (0,0);    
                \draw[line width=3pt, blue] (0,0) -- (0.7,0);  
            \end{tikzpicture}
            \textit{отложить прямую, равную данной прямой}
            \begin{tikzpicture}
                \node at (-0.7,0.2) {\small B};
                \node at (0.7,0.2) {\small C};
                \draw[line width=3pt] (-0.7,0) -- (0.7,0);    
            \end{tikzpicture}.
        \end{minipage}
        
        \vspace{1cm}
        
        \begin{center}
            Проведем 
            \begin{tikzpicture}
                \node at (-0.7,0.2) {\small A};
                \node at (0.7,0.2) {\small B};
                \draw[line width=3pt, densely dashed] (-0.7,0) -- (0.7,0);    
            \end{tikzpicture} (пост. I), \\
            \vspace{0.3cm}
            построим 
            \raisebox{-5ex}{
            \begin{tikzpicture}
                \node at (-0.7,1) {\small D};
                \node at (0.7,0.9) {\small A};
                \node at (0,-1) {\small B};
                \draw[line width=3pt, red] (-0.7,0.7) -- (0.7,0.6);  
                \draw[line width=3pt, densely dashed] (0.7,0.6) -- (0,-0.7);
                \draw[line width=3pt, densely dashed, red] (0,-0.7) -- (-0.7,0.7);
            \end{tikzpicture}} (пр. I.I), \\
            продлим 
            \begin{tikzpicture}
                \node at (-0.7,0.2) {\small B};
                \node at (0.7,0.2) {\small D};
                \draw[line width=3pt, densely dashed, red] (-0.7,0) -- (0.7,0);    
            \end{tikzpicture} (пост. II), \\
            опишем \raisebox{-3ex}{
            \begin{tikzpicture}
                \draw[blue, line width=3pt] (0,0) circle (0.7cm);
                \node at (0.1,0) {\small$B$};
                \node at (-1,0) {\small$C$};
                \draw[black,line width=3pt] (0,0) -- (-0.7,0);
            \end{tikzpicture}} (пост. III), 
            и \raisebox{-6ex}{
            \begin{tikzpicture}
                \draw[red, line width=3pt] (0,0) circle (1.1cm);
                \node at (0,0.2) {\small$D$};
                \filldraw[orange] (-60:1.1) circle (1pt) node[below right, text=black] {\small$E$};
                \draw[red,line width=3pt, densely dashed] (0,0) -- (-60:0.4);
                \draw[orange,line width=3pt] (-60:0.4) -- (-60:1.1);
            \end{tikzpicture}} (пост. III); \\
            продлим 
            \begin{tikzpicture}
                \node at (-0.7,0.2) {\small D};
                \node at (0.7,0.2) {\small A};
                \draw[line width=3pt, red] (-0.7,0) -- (0.7,0);    
            \end{tikzpicture} (пост. II), \\
            тогда искомая прямая это 
            \begin{tikzpicture}
                \node at (-0.7,0.2) {\small A};
                \node at (0.7,0.2) {\small F};
                \draw[line width=3pt, blue] (-0.7,0) -- (0.7,0);    
            \end{tikzpicture}.
            
            
            \vspace{1cm}
            
            Поскольку 
            \begin{tikzpicture}
                \node at (-0.7,0.2) {\small E};
                \node at (0.7,0.2) {\small D};
                \draw[line width=3pt, orange] (-0.7,0) -- (0.2,0);
                \draw[line width=3pt, densely dashed, red] (0.2,0) -- (0.7,0);    
            \end{tikzpicture} = 
            \begin{tikzpicture}
                \node at (-0.7,0.2) {\small D};
                \node at (0.7,0.2) {\small F};
                \draw[line width=3pt, red] (-0.7,0) -- (-0.2,0);
                \draw[line width=3pt, blue] (-0.2,0) -- (0.7,0);    
            \end{tikzpicture}(опр. 15),\\
            и 
            \begin{tikzpicture}
                \node at (-0.7,0.2) {\small B};
                \node at (0.7,0.2) {\small D};
                \draw[line width=3pt, densely dashed, red] (-0.7,0) -- (0.7,0);    
            \end{tikzpicture} = 
            \begin{tikzpicture}
                \node at (-0.7,0.2) {\small D};
                \node at (0.7,0.2) {\small A};
                \draw[line width=3pt, red] (-0.7,0) -- (0.7,0);    
            \end{tikzpicture} (постр.), \\
            $\therefore$ 
            \begin{tikzpicture}
                \node at (-0.7,0.2) {\small B};
                \node at (0.7,0.2) {\small E};
                \draw[line width=3pt, orange] (-0.7,0) -- (0.7,0);    
            \end{tikzpicture} = 
            \begin{tikzpicture}
                \node at (-0.7,0.2) {\small A};
                \node at (0.7,0.2) {\small F};
                \draw[line width=3pt, blue] (-0.7,0) -- (0.7,0);    
            \end{tikzpicture} (акс. III),\\
            но (опр. 15) 
            \begin{tikzpicture}
                \node at (-0.7,0.2) {\small B};
                \node at (0.7,0.2) {\small C};
                \draw[line width=3pt, black] (-0.7,0) -- (0.7,0);    
            \end{tikzpicture} = 
            \begin{tikzpicture}
                \node at (-0.7,0.2) {\small B};
                \node at (0.7,0.2) {\small E};
                \draw[line width=3pt, orange] (-0.7,0) -- (0.7,0);    
            \end{tikzpicture} = 
            \begin{tikzpicture}
                \node at (-0.7,0.2) {\small A};
                \node at (0.7,0.2) {\small F};
                \draw[line width=3pt, blue] (-0.7,0) -- (0.7,0);    
            \end{tikzpicture}.
            
            \vspace{1cm}
            
            $\therefore$ 
            \begin{tikzpicture}
                \node at (-0.7,0.2) {\small A};
                \node at (0.7,0.2) {\small F};
                \draw[line width=3pt, blue] (-0.7,0) -- (0.7,0);    
            \end{tikzpicture}, проведенная из данной точки \\
            (
            \begin{tikzpicture}
                \node at (0,0.2) {\small A};
                \draw[line width=3pt, red] (-0.7,0) -- (0,0);    
                \draw[line width=3pt, blue] (0,0) -- (0.7,0);  
            \end{tikzpicture}) равна данной прямой 
            \begin{tikzpicture}
                \node at (-0.7,0.2) {\small B};
                \node at (0.7,0.2) {\small C};
                \draw[line width=3pt] (-0.7,0) -- (0.7,0);    
            \end{tikzpicture} (акс. I).
            \vspace{0.3cm}
        \end{center}
        
        \begin{flushright}
        ч. т. д.
        \end{flushright}
    \end{minipage}
\end{document}