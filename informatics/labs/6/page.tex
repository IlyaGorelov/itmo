\begin{center}
\vspace{0.2cm}
\hrule
\vspace{0.3cm}
{\large\textbf{РЕЦЕНЗИИ. БИБЛИОГРАФИЯ}}
\vspace{0.2cm}
\hrule
\vspace{0.5cm}
\end{center}

{\huge \textbf{Всегда ли прав наш глаз?}}
\vspace{0.5cm}
\begin{multicols}{3}

{\centering \includegraphics[width=\linewidth]{pic.png}}

В разделе «Лаборатория Кванта» в 10-м и 11-м номерах нашего журнала за 1970 г. было рассказано о некоторых экспериментах со зрением. И цель их состояла в том, чтобы читатели поняли, что не всякое зрительное ощущение надо принимать как физическую реальность. Человеческий глаз — уникальный физический прибор, обладающий поразительной чувствительностью и точностью восприятия окружающего мира. Но и он в определенных условиях может совершать ошибки.

Знаменитый русский литературный герой — доктор X. В. Козьма Прутков советовал: «Если на клетке слона увидишь надпись “буйвол” — не верь глазам своим». Однако прежде глаза подводят нас даже там, где надпись сделана правильно.

В студенческие годы довелось делать серьёзный научный доклад на, казалось бы, совершенно анекдотическую тему: «О влиянии пения на зрение». Речь шла, в частности, о том, что глаза человека, напряженно ожидающего каких-то экспериментальных фактов, быстро устают и начинают видеть то, чего нет в действительности. Так вот, оказывается, музыка помогает восстановлению нормальной остроты и точности зрительных восприятий.

Так как глаз — важнейший «инструмент» физика, надо хорошо знать основные принципы его работы и границы его возможностей. Этому знакомству может существенно помочь книга Джеймса Грегга «Опыты со зрением в школе и дома», выпущенная в 1970 году издательством «Мир».

Книга эта содержит описание почти четырёх десятков опытов, которые, как правило, не требуют никакой специальной аппаратуры и при достаточной настойчивости вполне могут быть воспроизведены в домашней обстановке. Последовательно проводя эти опыты, можно узнать много интересных и порой неожиданных сведений о механизме зрительного восприятия окружающей действительности.

Многие из приведенных Греггом опытов характеризуют различные стороны глаза, рассматриваемые как оптическая система: отражение в глазном (опыт 2 и 4); живая

{\noindent\rule{0.5\linewidth}{0.4pt}} \\
 \ *)\textls[500]{Джеймс Грегг}, \\
Опыты со зрением в школе и дома. — «Мир», 1970, 197 стр.

\columnbreak 
 диафрагма глаза — зрачок (опыт 6); хроматическая аберрация оптической системы глаза (опыт 8); поле зрения (опыт 13); острота зрения (опыт 15) — вот некоторые из опытов, касающихся оптических свойств глаза. Это, так сказать, физика нашего зрения.

Процессы, благодаря которым мы видим окружающий мир, очень сложны и их нельзя понять без учета работы нашей нервной системы, то есть без исследования физиологии зрения. Лучи света, проходящие в глаз, раздражают окончания нервных волокон зрительного нерва. Эти сигналы поступают в наш мозг и во многом еще неполным образом вызывают картину увиденного. При этом мозг корректирует, подправляет информацию, полученную от наших глаз, посредством накопленных человеком опытов.

Видели ли вы, как нерешительные движения ребенка-малыша, еще не научившегося произносить первые слова? Как часто он пытается ставить кухонную игрушку совсем не так, где она действительно находится. А все потому, что мозг ребенка еще не научился помогать его глазам. На это нужно некоторое время.

Однако и у взрослых людей в так называемых малых зрениях существуют определенные ошибки и возможности, за пределами которых зрение начинает нас обманывать, потому что получаемая глазом информация оказывается недостоверной.
\end{multicols}

и проверим, что для каждого отдельного  $l$  суммы членов вида $X^{k-t}y^t$  в правой и левой частях равенства равны *) (коэффициент  $C_k^l$  мы не пишем):

\begin{center}
\begin{tabular}{|Sc|Sc|Sc|}
\hline
    & Числа с нечетной суммой цифр & Числа с четной суммой цифр \\
\hline
Сумма членов вида  $X^k$  & \[ 5 \cdot 10^{n-1} \sum A^k + 5 \cdot 10^{n-1} \sum B^k \] & \[ 5 \cdot 10^{n-1} \sum A^k + 5 \cdot 10^{n-1} \sum B^k \] \\
\hline
\makecell{Сумма членов вида $X^{k-l}y^l$ \\
$1\leq l\leq k-1$} & 
\makecell{
\[\sum A^{k-l} b^l + \sum B^{k-l} a^l = \\=
\sum b^l \sum A^{k-l} + \sum a^l \sum B^{k-l} =\\=
s_l \left( \sum A^{k-l} + \sum B^{k-l} \right)
\]} & 
\makecell{
\[\sum A^{k-l} a^l + \sum B^{k-l} b^l = \\
= \sum a^l \sum A^{k-l} + \sum b^l \sum B^{k-l} = \\
= s_l \left( \sum A^{k-l} + \sum B^{k-l} \right)\]} \\
\hline
Сумма членов вида $ y^k $ & \[ 5 \sum b^k + 5 \sum a^k \] & \[ 5 \sum a^k + 5 \sum b^k \] \\
\hline
\end{tabular}
\end{center}

Заметим, что нашу первоначальную выкладку для  $n=2$  с помощью аналогичных обозначений можно записать так:
 \[
\sum (A+b) + \sum (b+a) = 5 \sum A + 5 \sum a + 5 \sum b + 5 \sum a,
\quad
\sum (A+a) + \sum (B+b) = 5 \sum A + 5 \sum a + 5 \sum B + 5 \sum b
\]
при  $k=1$ остаются только первый и последний члены, соответствующие $l=0$ и $l=k$.

Нетрудно видеть, что утверждение задачи справедливо не только в десятичной, но и в любой другой системе счисления с основанием  $d$ , где $d$ — четное число (получайте, где в нашем решении используется четность основания  $d=10$). Если взять $ d=2 $ , получается такой любопытный ряд равенств:

\begin{center} 
\begin{gather*}
1+2=3 \\
1+2+4+7=3+5+6 \\
1^2+2^2+4^2+7^2=3^2+5^2+6^2\\
1+2+4+7+8+11+13+14=3+5+6+9+10+12+15 \\
1^2+2^2+4^2+7^2+8^2+11^2+13^2+14^2=3^2+5^2+6^2+9^2+10^2+12^2+15^2 \\
1^3+2^3+4^3+7^3+8^3+11^3+13^3+14^3=3^3+5^3+6^3+9^3+10^3+12^3+15^3 \\
. . . . . . . . . 
\end{gather*}
\end{center}

